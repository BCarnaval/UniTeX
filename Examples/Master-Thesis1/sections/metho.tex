\chapter{Groupe de renormalisation}
\begin{Abstract}
    \begin{changemargin}{2cm}{2cm}
    Le but de la première partie de ce chapitre est de donner toutes les clés
    de compréhension concernant le groupe de renormalisation wilsonien. On
    commencera par présenter certains concepts qui seront essentiels tout
    au long de cet ouvrage. Ensuite, je poserai les bases mathématiques
    nécessaires pour ensuite introduire le groupe de renormalisation. Enfin,
    la dépendence temporelle des constantes de couplage dans le calcul de
    renormalisation sera traité. Pour acquérir plus d'informations et
    d'intuition sur le sujet hors du présent texte, je conseille
    ceci\cite{ma-introduction-1973}.
    \end{changemargin}
\end{Abstract}

\section{contexte}

\section{Quelques définitions}
Définissons quelques concepts qui seront centrals tout au long de ce mémoire.
\begin{description}
    \item[adiabaticité:] À plusieurs reprises dans ce mémoire, je parlerai
    de régime adiabatique et non-adiabatique. un processus adiabatique
    est un processus par laquelle un système se transforme sans échanger
    de chaleur avec son environnement. Dans notre cas, on s'intérèsse aux
    interactions électrons-phonons. Lorsque l'on mentionne que l'on se trouve
    dans le régime \textit{adiabatique}, cela signifie que les électrons
    percoivent les phonons du système comme étant fixe ou ayant une constante
    de ressort infini. Il y a alors peu ou pas d'échange entre les 2 et c'est
    pour cette raison que l'on qualifie se régime d'\textit{adiabatique}.
    \item[pertinent/non-pertinent: ]
\end{description}

\subsection{Algèbre de Grassmann}
définition plus avantage

\subsection{Théorème de Wick}
voir annexe

\subsection{Théorème des graphes connexes}
voir annexe

\section{Groupe de Renormalisation (RG)}

\subsection{Fonction de partition}
Généralement le but est de calculer la fonction de partition
    \begin{align}
        Z = \tra{e^{-\beta H}}
    \end{align}
    \begin{figure}[H]
        \centering
        \begin{tikzpicture}
            \draw[thick,->] (-5,0) -- (5,0) node[below,pos = 0.95]{$k$};
            \draw[thick,->] (0,-4) -- (0,4) node[left,pos = 0.95]{$E$};
            \draw[thick] (-3,2) -- (-1,-2);
            \draw[thick] (1,-2) -- (3,2);
            \draw[thick,color = violet] (3,1.9)--(3,2.1);
            \draw[thick,color = violet] (2.9,1.7)--(2.9,1.9);
            \draw[thick,color = violet] (1,-1.9)--(1,-2.1);
            \draw[thick,color = violet] (1.1,-1.7)--(1.1,-1.9);
        \end{tikzpicture}
        \caption{Une image qui n'est pas terminé}
        \label{}
    \end{figure}
\subsection{G-ologie}
